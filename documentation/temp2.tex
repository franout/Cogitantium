

\documentclass[a4paper,11pt, titlepage, twoside]{article}

%Cedit a tots els estudiants futurs
\title{Plantilla Per a TFG-TFM ETSEIB}
\author{Andrea Serrano Costafreda i Bartomeu Costa Prats }
\date{gener 2019}

\usepackage[utf8]{inputenc}

% Language and font encodings
\usepackage[english]{babel} %spanish, english, ...
\usepackage{lipsum}
\usepackage[utf8]{inputenc}
\usepackage[T1]{fontenc}
\usepackage{parskip}
\setlength{\parskip}{4mm}
\setlength{\footskip}{60pt}
\setlength{\headheight}{15pt}

%% Sets page size and margins
\usepackage[a4paper,top=3cm, bottom=3.2cm, inner=3cm, outer=2cm, footnotesep=1cm, heightrounded]{geometry}


%% Useful packages

\usepackage{float}
\usepackage{verbatim}
\usepackage{amsmath}
\usepackage{systeme}
\usepackage{graphicx}
\usepackage{multirow}
\usepackage{caption}
\usepackage{subcaption}
\usepackage{wrapfig}
\usepackage[colorinlistoftodos]{todonotes}
\usepackage[colorlinks=true, allcolors=blue]{hyperref}
\usepackage{titlesec}
\usepackage{fancyhdr}
\usepackage[firstpage]{draftwatermark}
\usepackage{transparent}
\usepackage{textcomp} %podem escriure ``o'' amb la comanda \textdegree també € amb \texteuro
\usepackage[gen]{eurosym} %tb official en lloc de ``gen''
\usepackage[section]{placeins} %Evita que les figures saltin de secció
\usepackage{fancyref}
\usepackage{array}
\usepackage{longtable}
\usepackage{mathtools}
\usepackage{commath}
\usepackage{scrextend}%Indentar un bloc sencer
\usepackage{tgpagella}
\usepackage{dtk-logos}

%%Comentaris
\begin{comment}
Aquí podem posar tots els comentaris que calgui sense anar posant %
REFERENCIAT
\label{marker}, \ref{marker} and \pageref{marker} \footnote{footnote text}

\begin{addmargin}[esq]{dta}
El text d'aquí incrementa els seus marges segons ``esq'' i ``dta''
\end{addmargin}
\end{comment}

%%Comença pàgina nova per cada Secció SI hi ha alguna cosa escrita a la anterior

\newcommand{\sectionbreak}{\cleardoublepage}


%%Caps de pagina i peus (E/O (even/odd), L/C/R (left/center/right) y H/F (header/footer))
\fancyhf{}
\fancyhead[ER]{Memòria}
\fancyhead[OL]{Podeu posar el títol aquí}
\fancyhf[ELH, ORH]{pàg. \thepage}
%\fancyfoot[C]{}
\fancyfoot[EL, OR]{
\includegraphics[scale=0.2]{ETSEIB.png}}
\pagestyle{fancy}
\raggedbottom

\SetWatermarkText{\hspace{9mm}\transparent{0.15}\includegraphics[scale=2.5]{ETSEIB.png}}
\SetWatermarkAngle{0}
\SetWatermarkLightness{1}



\begin{document}
\renewcommand{\refname}{Bibliografia}
\begin{titlepage}
    {\centering
    {\Huge Treball de Fi de Grau}\\
    \vspace{5mm}
    {\Large \textbf{Grau en Enginyeria en Tecnologies Industrials (GETI)}}\\
    \vspace{20mm}
    \Huge \textbf{Títol}\\
    \vspace{10mm}
    \Huge\textbf{MEMÒRIA}\\
    \vspace{3mm}
    \Large\today\\  %Si e lloc del dia de la darrera edició es vol una data fixa, elimineu \today i poseu la data
    }
    \vspace{20mm}
    \hspace{2mm}
    \begin{tabular}{l@{ } l}
        \vspace{5mm}
        \Large \textbf{Autor:} & \Large{Nom i Cognoms} \\
        \vspace{5mm}
        \Large\textbf{Director:} & \Large{Nom i Cognoms}\\
        
         \Large\textbf{Convocatòria: } & \Large{data}\\
    \end{tabular}\par
    \vspace{10mm}
    {\centering
    \includegraphics[scale=0.3]{ETSEIB.png}\\
    {\Large Escola Tècnica Superior \\ d'Enginyeria Industrial de Barcelona}\\
    \vspace{3mm}
    \includegraphics[scale=0.4]{UPC_logo.PNG}
    \par
    }
    \end{titlepage}
%\maketitle

\clearpage
\thispagestyle{empty}
\null\newpage 
\pagenumbering{arabic}

\section*{Resum}
1 Pàgina. 
\lipsum[3-8]
\clearpage\null\newpage

\tableofcontents
\listoffigures


\section{Prefaci}
Per no repetir informació és millor referir-se a altres apartats \ref{introducció}.\par
I recorda, sempre és important citar a la bibliografia \cite{Etiqueta}.\\
La bibliografia ha d'estar ordenada, en teniu un exemple a la pàgina \pageref{biblio}

\begin{figure}[H] %Recorda fixar la posició de les figures si no
	\centering
	\includegraphics[width=0.35\textwidth]{ETSEIB.png}
	\caption{Una imatge del logo de l'ETSEIB}
	\label{fig:ETSEIB} %Si es posen les etiquetes de forma ordenada, la vida es més fàcil...
\end{figure}

%------------------------------------------
\section{Introducció}\label{introducció}
\subsection{Objectius}
\subsection{Abast del projecte}
Com hi ha coses escrites en aquest apartat, el següent començarà en una pàgina senar nova.
%------------------------------------------
\section{Apartats}

%------------------------------------------
\section{Pressupost}

\section{Impacte ambiental}

\section*{Conclusions}
 \addcontentsline{toc}{section}{Conclusions}
 
\section*{Agraïments}
 \addcontentsline{toc}{section}{Agraïments}
 Gracies

 
 %INICI BIBLIOGRAFIA
 
 \begin{thebibliography}{99}\label{biblio}
 \addcontentsline{toc}{section}{Bibliografia}
 
 \bibitem{Etiqueta} \textsc{Autors del Text}, {} 
 \textit{Títol de l'obra}, EDITORIAL i publicació\\ 
 \texttt{Consultat \emph{quotiens necesse est}}
 

 \end{thebibliography}
 
 
\end{document}
