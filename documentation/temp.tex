\documentclass[12pt,oneside]{article}

%%%%%%%%%%%%%%%%%%%%%%%%%%%%
%%   Zusaetzliche Pakete  %%
%%%%%%%%%%%%%%%%%%%%%%%%%%%%
\usepackage{acronym}
\usepackage{enumerate}  
\usepackage{a4wide}
\usepackage{fancyhdr}
\usepackage{graphicx}
\usepackage{palatino}
\usepackage{blindtext}
\usepackage{multirow}
\usepackage[ruled,longend]{algorithm2e}
\usepackage{float}


%folgende Zeile auskommentieren für englische Arbeiten
\usepackage[ngerman]{babel}

\usepackage[T1]{fontenc}
\usepackage[utf8]{inputenc}
\usepackage[bookmarks]{hyperref}
\usepackage[justification=centering]{caption}
\usepackage[style=chicago-authordate,natbib=true,backend=biber]{biblatex}
\usepackage{csquotes}
\bibliography{literatur}

%%%%%%%%%%%%%%%%%%%%%%%%%%%%%%
%% Definition der Kopfzeile %%
%%%%%%%%%%%%%%%%%%%%%%%%%%%%%%

\pagestyle{fancy}
\fancyhf{}
\cfoot{\thepage}
\setlength{\headheight}{16pt}

%%%%%%%%%%%%%%%%%%%%%%%%%%%%%%%%%%%%%%%%%%%%%%%%%%%%%
%%  Definition des Deckblattes und der Titelseite  %%
%%%%%%%%%%%%%%%%%%%%%%%%%%%%%%%%%%%%%%%%%%%%%%%%%%%%%

\newcommand{\JMUTitle}[9]{

  \thispagestyle{empty}
  \vspace*{\stretch{1}}
  {\parindent0cm
  \rule{\linewidth}{.7ex}}
  \begin{flushright}
    \vspace*{\stretch{1}}
    \sffamily\bfseries\Huge
    #1\\
    \vspace*{\stretch{1}}
    \sffamily\bfseries\large
    #2
    \vspace*{\stretch{1}}
  \end{flushright}
  \rule{\linewidth}{.7ex}

  \vspace*{\stretch{1}}
  \begin{center}
    \includegraphics[width=2in]{siegel} \\
    \vspace*{\stretch{1}}
    \Large Seminararbeit/Bachelorarbeit/Masterarbeit  \\

    \vspace*{\stretch{2}}
   \large Juniorprofessur f\"{u}r Informationsmanagement\\
    \vspace*{\stretch{1}}
    \large Betreuer:  #7 \\[1mm]
    
    \vspace*{\stretch{1}}
    \large W\"urzburg, den #6
  \end{center}
}


%%%%%%%%%%%%%%%%%%%%%%%%%%%%
%%  Beginn des Dokuments  %%
%%%%%%%%%%%%%%%%%%%%%%%%%%%%

\begin{document}

  \JMUTitle
      {Titel }        % Titel der Arbeit
      {Autor}                        % Vor- und Nachname des Autors
      
      {Wirtschaftswissenschaftlichen Fakultät}  % Name der Fakultaet
      {W"urzburg 2018}                          % Ort und Jahr der Erstellung
      {dd.mm.yyyy}                              % Tag der Abgabe
      {Prof. Dr. Christian Janiesch}               % Name des Erstgutachters
      {Zweitgutachter}                          % Name des Zweitgutachters
      {Pr"ufungsdatum}                          % Datum der muendlichen Pruefung

  \clearpage

\lhead{}
\pagenumbering{Roman} 
    \setcounter{page}{1}

\tableofcontents
\clearpage

\addcontentsline{toc}{section}{\listfigurename}
\listoffigures

\addcontentsline{toc}{section}{\listtablename}
\listoftables
\clearpage

%%%%%%%%%%%%%%%%%%%%%%%%%%%%
%%  Kurzzusammenfassung   %%
%%%%%%%%%%%%%%%%%%%%%%%%%%%%
\markboth{Zusammenfassung}{Zusammenfassung}
\section*{Zusammenfassung}
\blindtext
\clearpage

%%%%%%%%%%%%%%%%%%%%%%%%%%%%
%%  Abstract   %%
%%%%%%%%%%%%%%%%%%%%%%%%%%%%
\markboth{Abstract}{Abstract}
\section*{Abstract}
\blindtext



%%%%%%%%%%%%%%%%%%%%%%%%%%%%
%%  Einstellungen  %%
%%%%%%%%%%%%%%%%%%%%%%%%%%%%
\cleardoublepage
\pagenumbering{arabic}  
    \setcounter{page}{1}
\lhead{\nouppercase{\leftmark}}

%%%%%%%%%%%%%%%%%%%%%%%%%%%%
%%  Hauptteil  %%
%%%%%%%%%%%%%%%%%%%%%%%%%%%%

\section{Einleitung} \label{einleitung}

Einleitungssatz für dieses Kapitel.

\subsection{Unterabschnitt}

\textbf{So wird dick geschrieben} und \textit{so kursiv}. \citet[580]{clemen1989combining} so wird Autor, Jahr und Seite zitiert. So wird in Klammern zitiert: \citep[580]{clemen1989combining} oder (\cites[580]{clemen1989combining}[548]{gilabert2006intelligent}). So wird eine Webquelle zitiert: \citet{shiny1}. So wird referenziert: Kapitel \ref{einleitung}, Gleichung \ref{eq:1} zeigt...

\begin{equation}
    \sum_{i=1}^N x_i
    \label{eq:1}
\end{equation}

Das ist eine Auflistung:

\begin{enumerate}
\item Element 1
\item Element 2
\end{enumerate}


%% remove blindtext
\blindtext

\subsection{Unterabschnitt zwei}

%% remove blindtext
\blindtext

\begin{figure}[H]
    \centering
    \includegraphics[width=0.3\textwidth]{siegel.pdf}
    \caption{Siegel der Universität}
    \label{fig:my_label}
\end{figure}

\subsubsection{UnterUnterabschnitt}

%% remove blindtext
\blindtext

\subsubsection{UnterUnterabschnitt zwei}
So schreibt man ein Algorithmus:
\\
\begin{algorithm}[H]
 \KwData{this text}
 \KwResult{how to write algorithm }
 initialization\;
 \While{not at end of this document}{
  read current\;
  \eIf{understand}{
   go to next section\;
   current section becomes this one\;
   }{
   go back to the beginning of current section\;
  }
 }
 \caption{How to write algorithms}
\end{algorithm}
\BlankLine


So gestaltet man eine Tabelle:
\newline
\begin{table}[H]
\caption{Beispielstabelle}
\centering
\begin{tabular}{llr}
\hline
A    & B & C \\
\hline
D      & per gram    & 11.65      \\
          & each        & 1.01       \\
E       & stuffed     & 32.54      \\
F       & stuffed     & 73.23      \\
G & frozen      & 8.39       \\
\hline
\end{tabular}
\end{table}

%%%%%%%%%%%%%%%%%%%%%%%%%%%%
%% Literaturverzeichnis wird 
%% automatisch eingefügt
%%%%%%%%%%%%%%%%%%%%%%%%%%%%
\clearpage
\lhead{}
\printbibliography
\addcontentsline{toc}{section}{\bibname}


%%%%%%%%%%%%%%%%%%%%%%%%%%%%
%% Eidesstattliche Erklärung
%% muss angepasst werden 
%% in Erklaerung.tex
%%%%%%%%%%%%%%%%%%%%%%%%%%%%
\input{Erklaerung.tex}

\end{document}
