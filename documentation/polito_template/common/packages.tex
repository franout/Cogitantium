%%%%%%%%%%%%%%%%%%%%%%%%%%%%%%%%%%%%%%%%%%%%%%%%%%%%
\usepackage[english]{babel}
\usepackage[utf8]{inputenc}
\usepackage[T1]{fontenc}
\usepackage{lmodern}

\usepackage{hyperref} % must be loaded before glossaries-extra

% bibliography
\usepackage[hyperref=true,backref=true,backend=biber,maxbibnames=9,maxcitenames=2,style=numeric,citestyle=numeric,sorting=none]{biblatex} % hyperref uses links, backref goes back to citations, uses biber as backend, with 9 names at most in bibliography and 2 in citations, citing using numbers, and sorting in citation order
% sorting can be also ydnt for year descending, name, title or ynt for ascending year

\usepackage{adjustbox} % to resize boxes by keeping the same aspect ratio
\usepackage{algorithm} % algorithm environment
\usepackage{algpseudocode} % improved pseudo-code
\usepackage{amsfonts}               %  AMS mathematical fonts
\usepackage{amsmath}
\usepackage{amssymb}                %  AMS mathematical symbols
\usepackage{bm}                     %  black/bold mathematical symbols
\usepackage{booktabs}               %  better tables
\usepackage[labelfont=bf]{caption} % font=footnotesize % to have reduced caption font size
\usepackage{csquotes}
\usepackage{enumitem} %left align the bulleted points
\usepackage{geometry}
%\usepackage{glossaries} % to use acronyms and glossary, it has also glossaries-extra as extension, but commands are different
\usepackage[%
    toc, % puts the link in the ToC
    %record, % to use bib2gls
    abbreviations, % to load abbreviations / acronyms
    nonumberlist, % to avoid printing the numbers of the references in the acronyms page
]{glossaries-extra}
\usepackage{graphicx}               %  post-script images
%\usepackage{iwona} % extra fonts, substitute standard ones
\usepackage{listings} % to insert formatted code
\usepackage{lipsum} % for lorem ipsum text, not needed in the real work
\usepackage{makecell} % to change dimensions of cells, for math cases
\usepackage{mathtools} % for additional commands
\usepackage{mfirstuc} % to have capitalization capabilities
\usepackage[final]{microtype}      % microtypography, final lets latex use it also in bibliography
\usepackage{multirow} % to allow for cells covering more than 1 row in tables
\usepackage{nicefrac}       % compact symbols for 1/2, etc.
%\usepackage[lofdepth,lotdepth]{subfig}
\usepackage{ragged2e} % for justifying text
\usepackage{siunitx} % support for SI units of measurement and number typesetting
\usepackage{subfig}
\usepackage{svg} % for svg support, works only if inkscape is installed, default for Overleaf v2
%\usepackage{subfigure}              %  subfigure compatibility, can be removed if subfig
\usepackage{tabularx} % equal-width columns in tables
\usepackage{textcomp} % extra fonts and symbols
\usepackage{url}            % simple URL typesetting
\usepackage{verbatim} % for extended verbatim support
\usepackage{xcolor} % to define colors and use standard CSS names add dvipsnames as option, but it clashes with xcolor loaded in toptesi, pay attention that if it goes in conflict with tikz/beamer, simply use \documentclass[usenames,dvipsnames]{beamer}, along with other custom options when defining the document class

% \usepackage[backend=bibtex,  citestyle=numeric-comp,sorting=none,style=chem-angew ,articletitle=true]{biblatex}
 
 \usepackage{float}
 
 %for figure
 \usepackage{caption}


 
 \usepackage{epigraph}

\usepackage{tikz}
\usetikzlibrary{automata,positioning}

\usepackage{listings}

\usepackage{xcolor}


\lstdefinestyle{verilog-style}
{
    language=Verilog,
    basicstyle=\small\ttfamily,
    keywordstyle=\color{blue},
    identifierstyle=\color{black},
    commentstyle=\color{green},
    numbers=left,
    numberstyle=\tiny\color{black},
    numbersep=10pt,
    tabsize=8,
    moredelim=*[s][\colorIndex]{[}{]},
    literate=*{:}{:}1
}

\makeatletter
\newcommand*\@lbracket{[}
\newcommand*\@rbracket{]}
\newcommand*\@colon{:}
\newcommand*\colorIndex{%
    \edef\@temp{\the\lst@token}%
    \ifx\@temp\@lbracket \color{black}%
    \else\ifx\@temp\@rbracket \color{black}%
    \else\ifx\@temp\@colon \color{black}%
    \else \color{vorange}%
    \fi\fi\fi
}
\makeatother




\definecolor{veryLightGrey}{rgb}{0.9629411,0.9629411,0.9629411}
	
\lstset{language=C++,
        alsolanguage=[ANSI]C,
	keywordstyle=\color{vblue},
	keywordstyle=[2]\color{red},
	alsoletter={[2]\#},
	morekeywords={[2]\#include},
	basicstyle=\small\sffamily,
	commentstyle=\color{green}\small\sffamily,
	stringstyle=\sffamily,
	showstringspaces=false,
	breaklines=true,
	frame=none,
	numbers=left,                    % where to put the line-numbers; possible values are (none, left, right)
  numbersep=5pt,                   % how far the line-numbers are from the code
  numberstyle=\tiny\color{commentsColor}
  }


\setlength\epigraphwidth{10cm}
\setlength\epigraphrule{0pt}


%% for centering in the tables
\usepackage{array}
\newcolumntype{P}[1]{>{\centering\arraybackslash}p{#1}}



% Header and footer settings (Select TWOSIDE or ONESIDE layout below)
\usepackage{fancyhdr}								
\pagestyle{fancy}  
\renewcommand{\chaptermark}[1]{\markboth{\thechapter.\space#1}{}} 


% Select one-sided (1) or two-sided (2) page numbering
\def\layout{2}	% Choose 1 for one-sided or 2 for two-sided layout
% Conditional expression based on the layout choice
\ifnum\layout=2	% Two-sided
    \fancyhf{}			 						
	\fancyhead[LE,RO]{\nouppercase{ \leftmark}}
	\fancyfoot[LE,RO]{\thepage}
		\renewcommand{\footrulewidth}{0.4pt}
	\fancyfoot[C]{Politecnico di Torino}
	\fancyfoot[RE,LO]{\textit{Francesco Angione}}	
	\fancypagestyle{plain}{			% Redefine the plain page style
	\fancyhf{}
	\renewcommand{\headrulewidth}{0pt}
	\renewcommand{\footrulewidth}{0.4pt}
	\fancyfoot[LE,RO]{\thepage}
	\fancyfoot[C]{Politecnico di Torino}
	\fancyfoot[RE,LO]{\textit{Francesco Angione}}
	}	
	\fancypagestyle{plain_cover}{
	\fancyhf{}
	\renewcommand{\headrulewidth}{0pt}
		\renewcommand{\footrulewidth}{0pt}
	\fancyfoot[LE,RO]{\thepage}
	}
\else			% One-sided  	
  	\fancyhf{}					
	\fancyhead[C]{\nouppercase{ \leftmark}}
	\fancyfoot[C]{\thepage}
		\fancyfoot[C]{Chalmers University Of Technology}
	\fancyfoot[RE,LO]{\textit{Francesco Angione}}}	
\fi




