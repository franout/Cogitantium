% CREATED BY DAVID FRISK, 2016
\chapter{Conclusion}

\section{Discussion}A big portion of inference process for Neural Networks involves massive multiply and add computation, basic operation of tensor convolutions, and across several execution data, especially weight tensors, are reused.
As consequence, for speeding-up and reduce the power consumption (especially in mobile devices) of ML models an hardware accelerator has been developed.
It is also designed for accommodating different data type computation request from Neural Network models, ranging from integer8/16/32/64 to floating-point 32 and brain floating-point 16.\\\\

The approach of the work has been a hardware/software co-design in order to accommodate the high compute intensive request of Machine Learning, the tensor convolution. Therefore, the hardware core for tensor convolution has been developed from scratch, while the common components, such as memories and bus interface, have been chosen from the available ones in the tools.
Moving one step at the time above in the abstraction level, the accelerator library has been developed and deployed. In order to accomplish it in a fixed time, the core of the library has been developed in Python, which has been interfaced with a C-code template provided from the developers of thee ML-framework used. This has lead to a hybrid library which encapsulates a frozen Python code layer, called from the C-code, the latter is only in charge of retrieving the data and passing them to the Python layer.
Again moving one step above in the abstraction, the ML-framework level is reached. In this level, the most popular ML-framework, TensorFlow, has been chosen. It also offers the possibility of delegate part of the execution graph to coprocessor or GPUs. Moreover, existing Tensorflow pretrained models have been quantized for different bitwidth and data precision.\\\\

It is possible to build a custom hardware accelerator for a specific ML operation and then integrate it into a framework without changing the model nor the framework.
The bottom up approach and the delegate class available in Tensorflow has allowed to fully tailor a new class of hardware accelerators which can accommodate different needs (i.e. depending on which part of the model has to be accelerated). As it has been organized, changing the core software in the Python code and the core in the hardware, it can be also used for addressing different model's operations.

\section{Future Works}
For every human artifacts, there is always work to do. In addition, for Computer Engineering artifacts there is also an important step which is the software (and in this case also of the hardware) optimization.
In particular:\\
\begin{itemize}
\item Software Optimization and migration to a full C code implementation for further reducing the latency.
\item A deep software/hardware testing for finding additional bugs.
\item Power estimation using the simulation’s switching activity in order to obtain a very precise and reliable power consumption.
\item Comparison of model execution on different state-of-the-art platforms.\\
\end{itemize}

Following the previous recommendation, the work may arrive to a competitive level such as the one of the GPUs or other hardware platforms.
