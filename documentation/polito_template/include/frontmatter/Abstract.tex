% CREATED BY DAVID FRISK, 2016
%\oneLineTitle\\
%\oneLineSubtitle\\
%NAME FAMILYNAME\\
%Department of Computer Science and Engineering\\
%Chalmers University of Technology and University of Gothenburg\setlength{\parskip}{0.5cm}

%\thispagestyle{plain_cover}			% Supress header 
%\setlength{\parskip}{0pt plus 1.0pt}
%\section*{Abstract}
Part of a Neural Network inference execution mainly consists in multiplications and additions, basic operation of tensor convolutions, and across several execution data, especially weight tensors, are reused.
Clearly, those operations are executed on a CPU but, as it is well known, they are independent of each other and therefore they can be executed in parallel by the means of parallel architectures, such as GPU or domain specific hardware platform.
In the following pages, the state-of-the-art for accelerating Neural Network inference is explored starting from the newest proposed GPGPU architecture by NVIDIA to the domain specific accelerator from Google, NVIDIA, and Habana.\\
With the state-of-the-art awareness, a hardware accelerator capable of execution tensor convolution,  compute and memory intensive operation of a Neural Network, is designed from scratch. It is also designed for accommodating different data type computation request from Neural Network models, ranging from integer8/16/32/64 to floating-point 32 and brain floating-point 16.
Starting from the hardware system development, through the software development of a library capable to use the underlying hardware, it ends with integration into a popular Machine Learning framework, Tensorflow.\\
The work is carried out on a configurable hardware, FPGA, which allows to explore different design points, in terms of latency and number of processing elements, for different Neural Network models and data type. Moreover, the impact of integrating the accelerator into the Neural Network model is measured and compared with different platforms. Energy consumption is also estimated in the case of deployment on mobile devices.


% KEYWORDS (MAXIMUM 10 WORDS)
\vfill
Keywords: Computer, science, computer science, engineering, hardware, accelerator, machine learning.

\newpage				% Create empty back of side
\thispagestyle{empty}
\mbox{}