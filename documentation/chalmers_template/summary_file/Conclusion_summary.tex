% CREATED BY DAVID FRISK, 2016
\chapter{Conclusion}

\section{Discussion}A big portion of inference process for Neural Networks involves massive multiply and add computation, basic operation of tensor convolutions, and across several execution data, especially weight tensors, are reused.\\As consequence, for speeding-up and reduce the power consumption (especially in mobile devices) of ML models a hardware accelerator has been developed.
It is also designed for accommodating different data type computation request from Neural Network models, ranging from integer8/16/32/64 to floating-point 32 and brain floating-point 16.
It is possible to build a custom hardware accelerator for a specific ML operation and then integrate it into a framework without changing the model nor the framework.
The bottom up approach and the delegate class available in Tensorflow has allowed to fully tailor a new class of hardware accelerators which can accommodate different needs (i.e. depending on which part of the model has to be accelerated). As it has been organized, changing the core software in the Python code and the core in the hardware, it can be also used for addressing different model's operations.

\section{Future Works}
For every human artifacts, there is always work to do. In addition, for Computer Engineering artifacts there is also an important step which is the software (and in this case also of the hardware) optimization.
In particular:
\begin{itemize}
\item Software Optimization and migration to a full C code implementation for further reducing the latency.
\item A deep software/hardware testing for finding additional bugs.
\item Power estimation using the simulation’s switching activity in order to obtain a very precise and reliable power consumption.
\item Comparison of model execution on different state-of-the-art platforms.
\end{itemize}

Following the previous recommendation, the work may arrive to a competitive level such as the one of the GPUs or other hardware platforms.