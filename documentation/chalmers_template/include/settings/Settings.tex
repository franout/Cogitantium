% CREATED BY DAVID FRISK, 2016

% BASIC SETTINGS
\usepackage{moreverb}								% List settings
\usepackage{textcomp}								% Fonts, symbols etc.
\usepackage{lmodern}								% Latin modern font
\usepackage{helvet}									% Enables font switching
\usepackage[T1]{fontenc}							% Output settings
\usepackage[english]{babel}							% Language settings
\usepackage[utf8]{inputenc}							% Input settings
\usepackage{amsmath}								% Mathematical expressions (American mathematical society)
\usepackage{amssymb}								% Mathematical symbols (American mathematical society)
\usepackage{graphicx}								% Figures
\usepackage{subfig}									% Enables subfigures
\numberwithin{equation}{chapter}					% Numbering order for equations
\numberwithin{figure}{chapter}						% Numbering order for figures
\numberwithin{table}{chapter}						% Numbering order for tables


\usepackage{chemfig}		% Chemical structures
\usepackage[top=3cm, bottom=3cm,
			inner=3cm, outer=3cm]{geometry}			% Page margin lengths			
\usepackage{eso-pic}								% Create cover page background
\newcommand{\backgroundpic}[3]{
	\put(#1,#2){
	\parbox[b][\paperheight]{\paperwidth}{
	\centering
	\includegraphics[width=\paperwidth,height=\paperheight,keepaspectratio]{#3}}}}
\usepackage{float} 									% Enables object position enforcement using [H]
\usepackage{parskip}								% Enables vertical spaces correctly 
\usepackage{datetime} %date formatting tools

%\usepackage{minted} 		% Enables source code listings


% OPTIONAL SETTINGS (DELETE OR COMMENT TO SUPRESS)

% Disable automatic indentation (equal to using \noindent)
\setlength{\parindent}{0cm}                         


% Caption settings (aligned left with bold name)
\usepackage[labelfont=bf, textfont=normal,
			justification=justified,
			singlelinecheck=false]{caption} 		

		  	
% Activate clickable links in table of contents  	
\usepackage{hyperref}								
\hypersetup{colorlinks, citecolor=black,
   		 	filecolor=black, linkcolor=black,
    		urlcolor=black}


% Define the number of section levels to be included in the t.o.c. and numbered	(3 is default)	
\setcounter{tocdepth}{5}							
\setcounter{secnumdepth}{5}	


% Chapter title settings
\usepackage{titlesec}		
\titleformat{\chapter}[display]
  {\Huge\bfseries\filcenter}
  {{\fontsize{50pt}{1em}\vspace{-4.2ex}\selectfont \textnormal{\thechapter}}}{1ex}{}[]


% Header and footer settings (Select TWOSIDE or ONESIDE layout below)
\usepackage{fancyhdr}								
\pagestyle{fancy}  
\renewcommand{\chaptermark}[1]{\markboth{\thechapter.\space#1}{}} 


% Select one-sided (1) or two-sided (2) page numbering
\def\layout{2}	% Choose 1 for one-sided or 2 for two-sided layout
% Conditional expression based on the layout choice
\ifnum\layout=2	% Two-sided
    \fancyhf{}			 						
	\fancyhead[LE,RO]{\nouppercase{ \leftmark}}
	\fancyfoot[LE,RO]{\thepage}
		\renewcommand{\footrulewidth}{0.4pt}
%	\fancyfoot[C]{Chalmers University Of Technology}
	\fancyfoot[C]{Politecnico di Torino}
	\fancyfoot[RE,LO]{\textit{Francesco Angione}}	
	\fancypagestyle{plain}{			% Redefine the plain page style
	\fancyhf{}
	\renewcommand{\headrulewidth}{0pt}
	\renewcommand{\footrulewidth}{0.4pt}
	\fancyfoot[LE,RO]{\thepage}
%	\fancyfoot[C]{Chalmers University Of Technology}
	\fancyfoot[C]{Politecnico di Torino}
	\fancyfoot[RE,LO]{\textit{Francesco Angione}}
	}	
	\fancypagestyle{plain_cover}{
	\fancyhf{}
	\renewcommand{\headrulewidth}{0pt}
		\renewcommand{\footrulewidth}{0pt}
	\fancyfoot[LE,RO]{\thepage}
	}
\else			% One-sided  	
  	\fancyhf{}					
	\fancyhead[C]{\nouppercase{ \leftmark}}
	\fancyfoot[C]{\thepage}
%		\fancyfoot[C]{Chalmers University Of Technology}
		\fancyfoot[C]{Politecnico di Torino}
	\fancyfoot[RE,LO]{\textit{Francesco Angione}}}	
\fi


% Enable To-do notes
\usepackage[textsize=tiny]{todonotes}   % Include the option "disable" to hide all notes
\setlength{\marginparwidth}{2.5cm} 


% Supress warning from Texmaker about headheight
\setlength{\headheight}{15pt}		

%  use bibtex references 
%, 
 \usepackage[backend=bibtex,  citestyle=numeric-comp,sorting=none,style=chem-angew ,articletitle=true]{biblatex}
 
 \usepackage{float}
 
 %for figure
 \usepackage{caption}


 
 \usepackage{epigraph}

\usepackage{tikz}
\usetikzlibrary{automata,positioning}

\usepackage{listings}

\usepackage{xcolor}


\lstdefinestyle{verilog-style}
{
    language=Verilog,
    basicstyle=\small\ttfamily,
    keywordstyle=\color{blue},
    identifierstyle=\color{black},
    commentstyle=\color{green},
    numbers=left,
    numberstyle=\tiny\color{black},
    numbersep=10pt,
    tabsize=8,
    moredelim=*[s][\colorIndex]{[}{]},
    literate=*{:}{:}1
}

\makeatletter
\newcommand*\@lbracket{[}
\newcommand*\@rbracket{]}
\newcommand*\@colon{:}
\newcommand*\colorIndex{%
    \edef\@temp{\the\lst@token}%
    \ifx\@temp\@lbracket \color{black}%
    \else\ifx\@temp\@rbracket \color{black}%
    \else\ifx\@temp\@colon \color{black}%
    \else \color{vorange}%
    \fi\fi\fi
}
\makeatother




\definecolor{veryLightGrey}{rgb}{0.9629411,0.9629411,0.9629411}
	
\lstset{language=C++,
        alsolanguage=[ANSI]C,
	keywordstyle=\color{vblue},
	keywordstyle=[2]\color{red},
	alsoletter={[2]\#},
	morekeywords={[2]\#include},
	basicstyle=\small\sffamily,
	commentstyle=\color{green}\small\sffamily,
	stringstyle=\sffamily,
	showstringspaces=false,
	breaklines=true,
	frame=none,
	numbers=left,                    % where to put the line-numbers; possible values are (none, left, right)
  numbersep=5pt,                   % how far the line-numbers are from the code
  numberstyle=\tiny\color{commentsColor}
  }


\setlength\epigraphwidth{10cm}
\setlength\epigraphrule{0pt}


%% for centering in the tables
\usepackage{array}
\newcolumntype{P}[1]{>{\centering\arraybackslash}p{#1}}





